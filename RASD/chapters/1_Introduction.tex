\chapter{Introduction}

\section{Purpose}

% TODO: To introduce some background based on the Context.

% rising food demand
% climate change
% covid19 pandemic

Over 58 percent of rural Indian households rely on agriculture as their primary source of income, with 80 percent of them being smallholder farmers with less than 2 hectares of land. Furthermore, more than 20 percent of smallholder agricultural households live in poverty. Food consumption is anticipated to climb by 59 percent to 98 percent by 2050, according to the Harvard Business Review \cite{global_demand_for_food}.

With climate change posing a serious danger to agriculture, a complete overhaul of the system that transports food from farmers to our tables is required. On the top of that, the coronavirus pandemic created a huge disruption in food supply chains, exposing the weak parts of small-scale farmers as well as the significance of constructing resilient food systems.

Telengana is India's 11th biggest state, having a land area of 122,077$km^2$ and a population of 35,193,978 people (data from 2011) \cite{telengana}. \textbf{DREAM}'s goal is to help Telengana’s government promote data-driven policy-making in the state by designing, developing, and demonstrating anticipatory governance models for food systems utilizing digital public goods and community-centric approaches \cite{reference_doc}.

\subsection{Goals}
\begin{itemize}
    \item [\textbf{G1.}] Strengthen data-driven policy-making in Telengana.
    \item [\textbf{G2.}] Allow performance assessment of the farmers.
    \item [\textbf{G3.}] Ensure regular support of farmers by agronomists.
    \item [\textbf{G4.}] Allow farmers to exchange their knowledge.
\end{itemize}

\section{Scope}

Data dRiven PrEdictive FArMing (DREAM) is a project that intends to redesign the food production process in the Telengana area in order to create more resilient agricultural systems that will be required to meet the region's rising food demand. To address such a broad issue, it is critical to bring together individuals from many professions and backgrounds; hence, the participation of groups such as stakeholders, policy makers, farmers, analysts, and agronomists is desirable.

Data acquisition and combination can be viewed as the first feasible step in achieving the goal. This information might be gathered from farmers, specialized sensors installed on the ground that detect soil humidity, and government agronomists who visit fields on a regular basis.

The system aims to enable policy makers to identify farmers who perform well, especially under tough weather conditions, farmers who perform particularly badly and require assistance, and understand if agronomist-led steering initiatives yield substantial outcomes. This is going to be achieved by continuous assessment based on the visualizations of the acquired data.

Another group that would benefit from the system are the farmers, since it will allow them to exchange good practices between each other and request help when needed. All of that will be supplemented with visualizations of relevant data, including suggestions and weather forecasts.

The project also intends to facilitate the work of agronomists by creating a daily schedule for field visits, giving statistics on farmers' performance, and informing them about incoming assistance requests.

The system attempts to address a highly difficult topic, which necessitates the coordination of numerous individuals. All of these criteria requires the system to be very easy to use, intuitive, and straightforward, so that everyone can get the most out of it.

\subsection{World Phenomena}

\begin{itemize}
    \item [\textbf{WP}] Telengana collects information concerning short-term and long-term meteorological forecasts.
    \item [\textbf{WP}] Water irrigation systems collect information about the amount of water used by each farm
    \item [\textbf{WP}] Sensors collect information about the humidity of the soil at a given farm.
    \item [\textbf{WP}] An agronomist visits a farm.
    \item [\textbf{WP}] Farmer produces products on the farm.
    \item [\textbf{WP}] Farmer experiences issues on the farm.
\end{itemize}

\subsection{Shared Phenomena}

\begin{itemize}
    \item [\textbf{SP M}] The system reads information about short-term and long-term meteorological forecasts provided by Telengana API.
    \item [\textbf{SP M}] The system reads information about the amount of water used by each farm, provided by water irrigation systems API.
    \item [\textbf{SP M}] The system reads the data collected by the sensors at a given farm.
    \item [\textbf{SP W}] An agronomist saves information collected during a farm visit in the system.
    \item [\textbf{SP M}] The system notifies an agronomist about a scheduled visit in a farm.
    \item [\textbf{SP W}] Policy maker assesses farmer performance, giving him negative, neutral or positive note and saves it in the system.
    \item [\textbf{SP M}] The system shows a list of farmers.
    \item [\textbf{SP M}] The system shows farmer's summary.
    \item [\textbf{SP W}] A policy maker understands whether activities performed by an agronomists and farmers helped a farmer with a negative note using data from the system.
    \item [\textbf{SP M}] The system shows farmer's relevant data.
    \item [\textbf{SP M}] The system determines personalized suggestions concerning farmer.
    \item [\textbf{SP W}] Farmer inserts production data to the system.
    \item [\textbf{SP W}] Farmer request for help in the system.
    \item [\textbf{SP M}] The system sends a request for help to well performing farmers and agronomists in a given region. [happens in two cases: when a policy maker gives a negative note and when a farmer requests help]
    \item [\textbf{SP W}] Farmer creates a thread on the forum. 
    \item [\textbf{SP W}] Farmer comments on a given thread.
    \item [\textbf{SP W}] Agronomist inserts the area of responsibility into the system.
    \item [\textbf{SP M}] The system shows weather forecasts.
    \item [\textbf{SP M}] The system visualizes data about farmers with positive note in the area of responsibility for agronomist.
    \item [\textbf{SP M}] The system shows the daily plan of an agronomist.
    \item [\textbf{SP W}] Agronomist updates daily plan.
    \item [\textbf{SP W}] Agronomist confirms the execution of daily plan at the end of each day.
    \item [\textbf{SP W}] Agronomist rejects the execution of daily plan, while specifying deviations from the plan.
\end{itemize}

\section{Definitions, Acronyms, Abbreviations}

\subsection{Definitions}

\begin{center}
	\begin{tabular}{@{}p{0.30\linewidth} p{0.70\linewidth}@{}}
		\toprule
		\textbf{Expression}     & \textbf{Definition}\\
		\midrule
		Water irrigation system & External device, which collects information about the amount of water used on a given farm. Collected data are available via an API. \\
        Sensor                  & External device, which collects information about the humidity of soil on a given farm. Collected data are available via an API.\\
        Farmer's summary        & Information available to policy makers and agronomists, containing data about their production, help requests and responses, weather conditions, and information from water irrigation system and sensors.\\
        Farmer's note           & Evaluation note given by a governmental policy maker based on a farmer's performance. Can be negative, neutral, or positive.\\
        Personalized suggestion & Suggestion determined by the system. concerning specific crops to plant or specific fertilizers to use – based on farmers location and type of production.\\
        Farmer's relevant data  & Weather predictions, individualized recommendations on which crops to plant or which fertilizers to use – all depending on their location and kind of production.\\
        Production data         & Information provided by a farmer about his production. It lists product categories and quantities generated per product.\\
        Mandal                  & A local government area in India.\\
        Agronomist's area of responsibility  & Set of mandals that an agronomist is responsible for.\\
        Agronomist's daily plan & A daily schedule for visiting farms in a certain area of responsibility.\\
	\end{tabular}
\end{center}

\subsection{Acronyms}

\begin{center}
	\begin{tabular}{@{}p{0.25\linewidth} p{0.71\linewidth}@{}}
		\toprule
		\textbf{Acronyms}   & \textbf{Expression}\\
		\midrule
		DREAM               & Data-dRiven PrEdictive FArMing in Telengana\\
		RASD                & Requirements Analysis and Specifications Document\\
		G                   & Goal\\
		WP                  & World Phenomena\\
		SP                  & Shared Phenomena\\
		\bottomrule
	\end{tabular}
\end{center}

\subsection{Abbreviations}
\begin{center}
	\begin{tabular}{@{}p{0.25\linewidth} p{0.71\linewidth}@{}}
		\toprule
		\textbf{Abbreviations}  & \textbf{Expression}\\
		\midrule
	    text & text\\
		\bottomrule
	\end{tabular}
\end{center}

\section{Revision History}

\begin{center}
	\begin{tabular}{@{}p{0.18\linewidth} p{0.18\linewidth} p{0.57\linewidth}@{}}
		\toprule
		\textbf{Date}   & \textbf{Revision} & \textbf{Notes}\\
		\midrule
		\date{}         & v.1.0             & First release.\\
% 		\date{} & v.1.1 & \begin{itemize}[label={--},leftmargin=.4cm,noitemsep,topsep=0pt,before=\vspace{-3.5mm},after=\vspace{-4mm}]
% 			\item Change 1
% 			\item Change 2
% 		\end{itemize}\\
		\bottomrule
	\end{tabular}
\end{center}

\printbibliography[title={Reference Documents},keyword=intro, heading=subbibnumbered]

\section{Document Structure}

\begin{enumerate}
    \item \textbf{Introduction}: describes the general purpose of the project, as well as its scope. Furthermore, it defines specific definitions, acronyms, and abbreviations used throughout the document.
    \item \textbf{Overall Description}: provides a general description of the system, with the focus on its perspective, functions, characteristics, and assumptions.
    \item \textbf{Specific Requirements}: focuses on external interface, functional, and performance requirements. Design constraints and software system attributes are defined in this chapter. Specific use cases are depicted on \textit{UML} diagrams.
    \item \textbf{Formal Analysis using Alloy}: describes the formal modeling activity, and the model itself. Additionally, it provides verification of critical aspects of the system and an example of the generated world.
    \item \textbf{Effort Spent}: depicts information about the project contribution of each group member.
    \item \textbf{References}
\end{enumerate}
