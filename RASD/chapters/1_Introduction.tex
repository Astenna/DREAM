\chapter{Introduction}

\section{Purpose}

Telengana’s government wants to promote data-driven policymaking in the state by designing, developing, and demonstrating anticipatory governance models for food systems utilizing digital public goods and community-centric approaches.

\subsection{Goals}
\begin{itemize}
    \item Allow Telengana's policymakers to identify farmers who are performing well.
    \item Allow Telengana's policymakers to identify farmers who are performing particularly badly.
    \item Allow Telengana's policymakers to determine whether the steering initiatives carried out by agronomists produce significant results.
    \item Allow Farmers to visualize data relevant to them e.g. weather forecasts.
    \item Allow Farmers to insert data about their production and any problem they face.
    \item Allow Farmers to request help and suggestions by agronomists and other farmers.
    \item Allow Farmers to share hints and advices with each other.
    \item Allow Agronomists to insert the area of responsibility.
    \item Give Agronomists a possibility to receive information regarding help requests and answers to them.
    \item Allow Agronomists to see data visualizations concerning weather forecasts, as well as the best performing farmers in the area.
    \item Allow Agronomists to visualize and update a daily schedule for visiting farms in the area. All farms must be visited at least twice a year, but those that are underperforming should be visited more regularly, depending on the sort of difficulties they are experiencing.
    \item Allow Agronomists to confirm the execution of the daily plan at the end of each day, and additionally, to specify the deviations from the plan.
    
    \item Assess performance of the farmers in accordance to the external factors.
    \begin{itemize}
        \item Collect information about external factors - weather forecasts, irrigation system data and humidity of soil.
        \item Collect farmer's production data.
    \end{itemize}
    \item Allow the exchange of hints and problems.
    \begin{itemize}
        \item Allow farmers to receive help from agronomists.
        \item Allow farmers to exchange their knowledge.
    \end{itemize}
    
    
    
    \item Support poorly performing farmers.
    \begin{itemize}
        \item Share knowledge and good practices among farmers.
        \item Help agronomists in focusing on poorly performing farmers.
    \end{itemize}
    \item Collect knowledge regarding good practices among well performing farmers\dots
    \item Grant that every farm will be visited regularly by an agronomist. 
    \item Ensure that agronomists are focusing on farmers in need according to their area of responsibility (?)

        
\end{itemize}

\section{Scope}

\subsection{World Phenomena}

\begin{itemize}%[label=(\textbf{WP\arabic*})]
    \item kforfkor
    \item Test.
\end{itemize}

\subsection{Shared Phenomena}

% \begin{itemize}[label=\textbf{SP\arabic*}]
%     \item Test.
% \end{itemize}

\section{Definitions, Acronyms, Abbreviations}

\subsection{Definitions}

\begin{center}
	\begin{tabular}{@{}p{0.25\linewidth} p{0.71\linewidth}@{}}
		\toprule
		\textbf{Expression} & \textbf{Definition}\\
		\midrule
		Text1. & Text 2\\
        Text3. & Text 4\\
	\end{tabular}
\end{center}

\subsection{Acronyms}

\begin{center}
	\begin{tabular}{@{}p{0.25\linewidth} p{0.71\linewidth}@{}}
		\toprule
		\textbf{Acronyms} & \textbf{Expression}\\
		\midrule
		DREAM & Data-dRiven PrEdictive FArMing in Telengana\\
		\bottomrule
	\end{tabular}
\end{center}

\subsection{Abbreviations}
\begin{center}
	\begin{tabular}{@{}p{0.25\linewidth} p{0.71\linewidth}@{}}
		\toprule
		\textbf{Abbreviations} & \textbf{Expression}\\
		\midrule
		G & Goal\\
		WP & World Phenomena\\
		SP & Shared Phenomena\\
		\bottomrule
	\end{tabular}
\end{center}

\section{Revision History}

\begin{center}
	\begin{tabular}{@{}p{0.18\linewidth} p{0.18\linewidth} p{0.57\linewidth}@{}}
		\toprule
		\textbf{Date} & \textbf{Revision} & \textbf{Notes}\\
		\midrule
		\date{} & v.1.0 & First release.\\
		\date{} & v.1.1 & \begin{itemize}[label={--},leftmargin=.4cm,noitemsep,topsep=0pt,before=\vspace{-3.5mm},after=\vspace{-4mm}]
			\item Change 1
			\item Change 2
		\end{itemize}\\
		\bottomrule
	\end{tabular}
\end{center}

\section{Reference Documents}

\begin{itemize}
    \item Assignment RDD AY 2021-2022 ("Requirement Engineering and Design Project: goal, schedule and rules")
\end{itemize}

\section{Document Structure}

\begin{enumerate}
    \item \textbf{Introduction}:
    \item \textbf{Overall Description}:
    \item \textbf{Specific Requirements}:
    \item \textbf{Formal Analysis using Alloy}:
    \item \textbf{Effort Spent}:
    \item \textbf{References}:
\end{enumerate}
