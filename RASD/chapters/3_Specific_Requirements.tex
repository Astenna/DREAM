\chapter{Specific Requirements}

\section{External Interface Requirements}

\subsection{User Interfaces}

\subsection{Hardware Interfaces}
DREAM is a web application, as such each user should own a device capable of installing a modern browser. Form factor of the device doesn't matter, as DREAM application is fully responsive.

\subsection{Software Interfaces}
A modern browser is necessary to use the system. DREAM supports most major browsers. However, in order to ensure full functionality, below is a list of browsers that are fully supported.
\begin{itemize}
    \setlength\itemsep{0em}
    \item Google Chrome
    \item Firefox
    \item Safari
    \item Microsoft Edge
\end{itemize}
User should always update their browser to keep up with security, compatibility and functionality.

\subsection{Communication Interfaces}
To perform any kind of operation using the system, a stable internet connection is required (WiFi or cellular).

\section{Functional Requirements}
% Definition of use case diagrams, 
% use cases and associated sequence/activity diagrams, 
% and mapping on requirements

\begin{table}[H]
    \centering
	\begin{tabular}{@{}p{0.25\linewidth} p{0.72\linewidth}@{}}
		\toprule
		\textbf{Name}               & Receive request notification\\
		\midrule
		\textbf{Id}                 & TODO\\
		\midrule
		\textbf{Actors}             & Agronomist, Farmer\\
		\midrule
		
		\textbf{Entry conditions}   & \begin{itemize}[leftmargin=.4cm,noitemsep,topsep=0pt,before=\vspace{-3mm},after=\vspace{-4mm}]
		    \item The farmer or the agronomist is already logged in to the application
		\end{itemize}\\
		\midrule
		
		\textbf{Flow of events}     & \begin{enumerate}[leftmargin=.4cm,noitemsep,topsep=0pt,before=\vspace{-3mm},after=\vspace{-4mm}]
		    \item The application indicates that unread notifications exist via a small number on the bell icon on the top bar.
		    \item The farmer or the agronomist clicks on the icon.
		    \item The application shows a list of unread notifications.
		    \item The farmer or the agronomist selects the unread notification to see.
		    \item The application shows details of the request.
		\end{enumerate}\\
		\midrule
		\textbf{Exit conditions}    & The application showed the notification contents. \\
		\midrule
		
		\textbf{Exceptions}         & \begin{itemize}[leftmargin=.4cm,noitemsep,topsep=0pt,before=\vspace{-3mm}]
		   \item The farmer or the agronomist cancels the operation during step 3.
		\end{itemize}
	    The system hides the list of unread notifications and shows only the main board screen.\\
		\bottomrule
	\end{tabular}
	\caption{Your caption here} 
\end{table}


\begin{center}
	\begin{tabular}{@{}p{0.25\linewidth} p{0.72\linewidth}@{}}
		\toprule
		\textbf{Name}               & Answer to a request\\
		\midrule
		\textbf{Id}                 & TODO\\
		\midrule
		\textbf{Actors}             & Agronomist, Farmer\\
		\midrule
		
		\textbf{Entry conditions}   & \begin{itemize}[leftmargin=.4cm,noitemsep,topsep=0pt,before=\vspace{-3mm},after=\vspace{-4mm}]
		    \item The farmer or the agronomist is already logged in to the application
		    \item The farmer or the agronomist has already read the notification
		\end{itemize}\\
		\midrule
		
		\textbf{Flow of events}     & \begin{enumerate}[leftmargin=.4cm,noitemsep,topsep=0pt,before=\vspace{-3mm},after=\vspace{-4mm}]
		    \item The farmer or the agronomist clicks \textit{Reply} under the request contents.
		    \item The farmer or the agronomist enters the contents of the reply message.
		    \item The farmer or the agronomist clicks the \textit{Send} button.
		    \item The application shows a message indicating successful sending of the message.
		\end{enumerate}\\
		\midrule
		\textbf{Exit conditions}    & The application notifies a farmer about the reply to the request. \\
		\midrule
		
		\textbf{Exceptions}         & \begin{itemize}[leftmargin=.4cm,noitemsep,topsep=0pt,before=\vspace{-3mm}]
		   \item The farmer or the agronomist cancels the operation during step 2.
		\end{itemize}
	    The system comes back to the screen with the list of notifications.\\
		\bottomrule
	\end{tabular}
\end{center}

\begin{center}
	\begin{tabular}{@{}p{0.25\linewidth} p{0.72\linewidth}@{}}
		\toprule
		\textbf{Name}               & Update daily plan\\
		\midrule
		\textbf{Id}                 & TODO\\
		\midrule
		\textbf{Actors}             & Agronomist\\
		\midrule
		
		\textbf{Entry conditions}   & \begin{itemize}[leftmargin=.4cm,noitemsep,topsep=0pt,before=\vspace{-3mm},after=\vspace{-4mm}]
		    \item The agronomist is already logged in to the application.
		\end{itemize}\\
		\midrule
		
		\textbf{Flow of events}     & \begin{enumerate}[leftmargin=.4cm,noitemsep,topsep=0pt,before=\vspace{-3mm},after=\vspace{-4mm}]
		    \item The agronomist selects the daily plan tab.
		    \item The agronomist selects a visit he wants to rearrange.
		    \item  The agronomist selects a new date for the visit selected in the previous step.
		    \item The agronomist confirms his choice.
		    \item The application shows a message indicating successful save of the visit's date.
		\end{enumerate}\\
		\midrule
		\textbf{Exit conditions}    & The new date of the visit is successfully saved in the application. \\
		\midrule
		
		\textbf{Exceptions}         & \begin{itemize}[leftmargin=.4cm,noitemsep,topsep=0pt,before=\vspace{-3mm}]
		   \item An agronomist cancels the operation during the step 3. \todo{add another exception if a visit can't be delayed anymore?}
		\end{itemize}
	    The system shows the daily plan tab view.\\
		\bottomrule
	\end{tabular}
\end{center}

\begin{center}
	\begin{tabular}{@{}p{0.25\linewidth} p{0.72\linewidth}@{}}
		\toprule
		\textbf{Name}               & Save farm visit information\\
		\midrule
		\textbf{Id}                 & TODO\\
		\midrule
		\textbf{Actors}             & Agronomist\\
		\midrule
		
		\textbf{Entry conditions}   & \begin{itemize}[leftmargin=.4cm,noitemsep,topsep=0pt,before=\vspace{-3mm},after=\vspace{-4mm}]
		    \item The agronomist is already logged in to the application.
		    \item The agronomist visited a farm.
		\end{itemize}\\
		\midrule
		
		\textbf{Flow of events}     & \begin{enumerate}[leftmargin=.4cm,noitemsep,topsep=0pt,before=\vspace{-3mm},after=\vspace{-4mm}]
		    \item The agronomist selects the daily plan tab.
		    \todo{is it a tab? do wee need to define it?}
		    \item The agronomist selects the visit to a farm he had already done. 
		    \item The agronomist enters information he collected during a farm visit to the text box.
		    \item The agronomist saves the information he entered.
		    \item The application shows a message indicating successful save of entered data.
		\end{enumerate}\\
		\midrule
		\textbf{Exit conditions}    & The information obtained during a visit to a farm is saved in the system. \\
		\midrule
		
		\textbf{Exceptions}         & \begin{itemize}[leftmargin=.4cm,noitemsep,topsep=0pt,before=\vspace{-3mm}]
		    \item An agronomist cancels the operation during the step 3.
		\end{itemize}
	    The system shows the daily plan tab view. 
	    \todo{the same as above}\\
		\bottomrule
	\end{tabular}
\end{center}

\section{Performance Requirements}

\section{Design Constraints}

\subsection{Standards compliance}

\subsection{Hardware limitations}

\subsection{Any other constraints}

\section{Software System Attributes}

\subsection{Reliability}

\subsection{Availability}

\subsection{Security}

\subsection{Maintainability}

\subsection{Portability}

\section{Formal Analysis Using Alloy}

\section{Effort Spent}

\section{References}