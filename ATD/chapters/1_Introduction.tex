\chapter{Introduction}

The focus of this document is to present the work done during testing of the Data-dRiven PrEdictive FArMing in Telangana (\textit{DREAM}) application's initial prototype, implemented by another group of students. The project was carried out in accordance with the directives and instructions outlined in the \textit{I\&T assignment goal, schedule, and rules} \cite{reference_doc2}. The \textit{Requirement Engineering and Design Project: goal, schedule, and rules} document \cite{reference_doc} provides a full background as well as a thorough overview of the problem.

\section{Analyzed Project}

The analyzed code was developed by Marco Domenico Buttiglione, Martina Caffagnini, and Dounia Faouzi and is available in their private repository hosted on GitHub \cite{dream}.

The most important works referenced when performing the acceptance testing were \textit{Requirements Analysis and Specifications Document} \cite{rasd}, \textit{Design Document} \cite{dd_doc}, and \textit{Implementation and Test Deliverable} \cite{itd_doc} written by the abovementioned people.

\section{Definitions and Acronyms}

\subsection{Definitions}

\begin{center}
    \begin{longtable}{@{}p{0.28\linewidth} p{0.68\linewidth}@{}}
		\toprule
		\textbf{Expression}     & \textbf{Definition}\\
		
		\midrule
        Example Expression      & Example definition.\\
	\bottomrule
	\end{longtable}
\end{center}

\subsection{Acronyms}

\begin{center}
	\begin{longtable}{@{}p{0.28\linewidth} p{0.68\linewidth}@{}}
		\toprule
		\textbf{Acronyms}   & \textbf{Expression}\\
		\endfirsthead
		\midrule
		API                 & Application Programming Interface\\
		DD                  & Design Document\\
		DREAM               & Data-dRiven PrEdictive FArMing in Telangana\\
		ITD					& Implementation and Test Deliverable\\
		RASD                & Requirements Analysis and Specifications Document\\
		SQL                 & Structured Query Language\\
		UI                  & User Interface\\
		\bottomrule
	\end{longtable}
\end{center}

\section{Revision History}

\begin{center}
	\begin{longtable}{@{}p{0.18\linewidth} p{0.18\linewidth} p{0.57\linewidth}@{}}
		\toprule
		\textbf{Date}   & \textbf{Revision} & \textbf{Notes}\\
		\midrule
        \todo{Specify date}     	& v.1.0             & First release.\\
		\bottomrule
	\end{longtable}
\end{center}

\printbibliography[title={Reference Documents}, keyword=intro, heading=subbibnumbered]
