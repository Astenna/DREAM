\chapter{Installation Setup} \label{ch:installation_setup}

This chapter describes in detail the steps performed to install and successfully run the initial prototype of the \textit{DREAM} application provided. The system consists of two different parts: backend and frontend, which were installed separately.

\section{Backend}

A database and a server constituted together the backend of the system. The database was created using the \textit{MySQL} database management system \cite{mysql}, whereas the server run on the \textit{Java Virtual Machine (JVM)} and was implemented in the \textit{Spring Boot} framework \cite{spring_boot}.

\subsection{Server Prerequisites}

\begin{enumerate}
    \item Downloaded backend code from the \href{https://github.com/marticaffa/CaffagniniFaouziButtiglione/tree/main/Code/back-end}{backend repository}. Nonetheless, it \textbf{was not} prepared in the form of a \textit{.zip} file as specified by the assignment document.
    \item Installed \textit{IntelliJ Ultimate} from the \href{https://www.jetbrains.com/idea/download/}{official website}.
    \item Allowed the IDE to resolve all the dependencies of the backend code.
    \item Built the code.
\end{enumerate}

\subsection{Database Prerequisites}

\begin{enumerate}
    \item Installed \textit{MySQL Sever \& Workbench} from the \href{https://dev.mysql.com/downloads/installer/}{official website} by accepting only community \textit{Server}, \textit{Workbench}, and \textit{Connector/J} packages.
    \item Set the default username and password for \textit{MySQL} server to \textit{"root"}. However, this \textbf{did not work} since it was not set to \textit{"root"} inside \textit{application.properties} file, which should be the case according to the instruction provided in the group's ITD document. Instead, it was set to \textit{"6EG7DGkcfH8pv2"}. This required the user to manually change the password to \textit{"root"} in the aforementioned file.
    \item Created database schema called \textit{db\_dream}.
\end{enumerate}

\subsection{Backend Setup}

\begin{enumerate}
    \item Run \textit{DreamApplication.java} inside \textit{IntelliJ Ultimate}.
    \item Executed \textit{db\_initialization.sql} inside \textit{MySQL Workbench}.
\end{enumerate}

\section{Frontend}

The frontend of the system was developed using \textit{JavaScript} together with \textit{ReactJS} \cite{reactjs}.

\subsection{Basic Setup}

Opened the following links inside a web browser: \href{https://dreamfarmerse2.firebaseapp.com}{farmer's app}, \href{https://dreamagronomistse2.firebaseapp.com}{agronomist's app}. This setup was utilized to perform the acceptance testing.

\subsection{Setup Used for Code Verification Purpose}

The following setup (see the listings \ref{lst:agronomist_web_app} and \ref{lst:farmer_web_app}) was employed to manually build the source code, and thereby, to verify its correctness. The source code was downloaded from the \href{https://github.com/marticaffa/CaffagniniFaouziButtiglione/tree/main/Code/UI}{frontend repository}. It is important to notice that there was no instruction provided inside the group's ITD that would help with this process.

\subsubsection*{Agronomist's WebApplication}

\begin{lstlisting}[language=bash, caption={Install dependencies and build \textit{Agronomist's WebApplication}.}, label=lst:agronomist_web_app]
cd Code/UI/dream-agronomist-webapp/
npm install
npm start
\end{lstlisting}

\subsubsection*{Farmer's WebApplication}

\begin{lstlisting}[language=bash, caption={Install dependencies and build \textit{Farmer's WebApplication}.}, label=lst:farmer_web_app]
cd Code/UI/dream-farmer-webapp/
npm install
npm start
\end{lstlisting}