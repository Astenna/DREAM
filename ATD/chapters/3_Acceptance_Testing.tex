\chapter{Acceptance Testing} \label{ch:acceptance_testing}

The following chapter reminds all project details that are considered to be valuable to design the test cases. Furthermore, it elaborates in detail on how the verification was performed, showing all the scenarios and their results.

\section{Project Specifics}

Goals, use cases, as well as functional requirements that are defined throughout the RASD document are listed below. They are further exploited to define particular test cases in the process of acceptance testing. Since the functionalities were implemented to satisfy only the needs of agronomists and farmers, all the information related to the role of a policy maker is crossed out.

\subsection{Goals}

\begin{enumerate}
	\item [\textbf{G1.}] Allowing farmer to share their knowledge and problems with other farmers and help them with their farm by creating and commenting a topic.
	\item [\textbf{G2.}] Allowing farmer to keep track of their production.
	\item [\textbf{G3.}] Showing the farmer personalized advice regarding specific subjects of their need, shared by the agronomist.
	\item [\textbf{G4.}] Allowing farmers to make a request for help to agronomists.
	\item [\textbf{G5.}] Allowing agronomists to help farmers in their area by answering to their request.
	\item [\textbf{G6.}] Allowing agronomists to help farmers in their area by sharing their knowledge and giving them advice.
	\item [\textbf{G7.}] Allowing agronomists to plan visits with farmers to check their progress and, if needed, help them.
	\item [\textbf{G8.}] \sout{Allowing policy makers to monitor the performance of the farmers to help them or to give them special incentives to encourage them to share their experience.} (\textbf{Not implemented.})
	\item [\textbf{G9.}] Showing customers weather forecasts allowing them to have a better approach with the climate changes.
\end{enumerate}

\subsection{Use Cases}

\begin{enumerate}
	\item [\textbf{U1.}] Show Farmer's Home Page.
	\item [\textbf{U2.}] Show Agronomist's Home Page.
	\item [\textbf{U3.}] Show Topic Page.
	\item [\textbf{U4.}] Show Production.
	\item [\textbf{U5.}] Show Farm Info.
	\item [\textbf{U6.}] Send Request for Help.
	\item [\textbf{U7.}] Sign Up.
	\item [\textbf{U8.}] Log In.
	\item [\textbf{U9.}] Create Meeting
	\item [\textbf{U10.}] Create Knowledge.
	\item [\textbf{U11.}] Insert Farm Data.
	\item [\textbf{U12.}] Notification New Farmer.
	\item [\textbf{U13.}] \sout{Show Policy Maker's Home Page.} (\textbf{Not implemented.})
	\item [\textbf{U14.}] Create New Topic.
	\item [\textbf{U15.}] Update Production Data.
	\item [\textbf{U16.}] Comment Topic.
	\item [\textbf{U17.}] Answer Help Request.
\end{enumerate}

\subsection{Requirements}

\begin{enumerate}
	\item [\textbf{R1.}] The system allows the farmer to add data concerning his production.
	\item [\textbf{R2.}] The system allows the agronomist to view the data concerning their farmers' production.
	\item [\textbf{R3.}] The system allows the agronomist to book an appointment with the farmer by showing all available slots.
	\item [\textbf{R4.}] The system allows the agronomist to modify a booked appointment with the farmer.
	\item [\textbf{R5.}] The system allows the agronomist to cancel a booked appointment with a farmer.
	\item [\textbf{R6.}] The system check that the agronomist booked at least two appointments per year with each farmer.
	\item [\textbf{R7.}] The system alerts the farmer before the appointment.
	\item [\textbf{R8.}] The system alerts the agronomist before the appointment.
	\item [\textbf{R9.}] The system allows the agronomist to view the calendar.
	\item [\textbf{R10.}] The system allows the farmer to view the calendar.
	\item [\textbf{R11.}] The system allows the farmer to send a help request to the agronomist.
	\item [\textbf{R12.}] The system allows the agronomist to answer to the help requests that are sent to him.
	\item [\textbf{R13.}] The system allows the agronomist to view all the help requests sent to him.
	\item [\textbf{R14.}] The system allows the farmer to view all the help requests sent by him.
	\item [\textbf{R15.}] The system allows the farmer to publish a topic on the forum.
	\item [\textbf{R16.}] The system allows all farmers to view the forum.
	\item [\textbf{R17.}] The system allows all farmers to publish an answer to a topic on the forum.
	\item [\textbf{R18.}] The system allows the farmers to check the weather forecasts for their area.
	\item [\textbf{R19.}] The system allows the agronomists to share knowledge with the farmer.
	\item [\textbf{R20.}] The system allows the farmers to visualize the \textit{knowledges} shared by their agronomist.
	\item [\textbf{R21.}] The system allows all agronomists to check the weather forecasts for their area.
	\item [\textbf{R22.}] The system allows the agronomists to publish a grade for each farmer after each visit.
	\item [\textbf{R23.}] \sout{The system allows the policy makers to view the grades of all farmers of Telangana.} (\textbf{Not implemented.})
	\item [\textbf{R24.}] The system requires a sign up and a login to access to the data.
	\item [\textbf{R25.}] The system shows to each agronomist the list of all farmers they are responsible for.
	\item [\textbf{R26.}] The system notifies the agronomists when a new farmer has registered under their area.
	\item [\textbf{R27.}] The system allows the farmer to insert their position when they first register.
\end{enumerate}

\section{Test Cases} \label{sec:test_cases}

The following test cases are used in the process of acceptance testing of the initial prototype of the DREAM system. They are derived from project's use cases, which are additionally mapped to the functional requirements so that the fulfillment of both is verified. All the testing was conducted manually.

Before starting the acceptance tests, the application was seeded with data required for specified test cases. Two farmer's accounts, help requests together with responses, forum topics and comments, products and their production data, as well as meetings and knowledge items were created.

\subsection{Show Farmer's Home Page}

\begin{longtable}{@{}p{0.25\linewidth}p{0.71\linewidth}@{}}
	\toprule
	\textbf{Use case} & \textbf{U1.} Show Farmer's Home Page\\
	\midrule
	\textbf{Goals} & G1, G2, G3, G4, G9\\
	\midrule
	\textbf{Requirements} & R7, R18, R20\\
	\midrule
	\textbf{Actor} & Farmer\\
	\midrule
	\textbf{Scenario} & 
	\begin{enumerate}[leftmargin=.4cm,noitemsep,topsep=0pt,before=\vspace{-3mm},after=\vspace{-4mm}]
		\item Open \textit{Farmer's WebApplication}.
		\item Click the \textit{Sign In} button.
		\item Fill in \textit{E-Mail} and \textit{Password} fields with credentials used in the \textit{Sign Up} process.
		\item Press the \textit{Sign In} button.
	\end{enumerate}\\
	\midrule
	\textbf{Expected Output} & The \textit{Home Page} with weather forecast, knowledge, and planned meetings is displayed.\\
	\midrule
	\textbf{Test Result} & Success\\
	\bottomrule
	\caption{Test case verifying U1.}
\end{longtable}

\subsection{Show Agronomist's Home Page}

\begin{longtable}{@{}p{0.25\linewidth}p{0.71\linewidth}@{}}
	\toprule
	\textbf{Use case} & \textbf{U2.} Show Agronomist's Home Page\\
	\midrule
	\textbf{Goals} & G5, G6, G7, G9\\
	\midrule
	\textbf{Requirements} & R8, R13, R21\\
	\midrule
	\textbf{Actor} & Agronomist\\
	\midrule
	\textbf{Scenario} & 
	\begin{enumerate}[leftmargin=.4cm,noitemsep,topsep=0pt,before=\vspace{-3mm},after=\vspace{-4mm}]
		\item Open \textit{Agronomist's WebApplication}.
		\item Click the \textit{Sign In} button.
		\item Fill in \textit{E-Mail} and \textit{Password} fields with credentials used in the backend setup process.
		\item Press the \textit{Sign In} button.
	\end{enumerate}\\
	\midrule
	\textbf{Expected Output} & The \textit{Home Page} with weather forecast, help requests, and planned meetings is displayed.\\
	\midrule
	\textbf{Test Result} & Success\\
	\bottomrule
	\caption{Test case verifying U2.}
\end{longtable}

\subsection{Show Topic Pane}

\begin{longtable}{@{}p{0.25\linewidth}p{0.71\linewidth}@{}}
	\toprule
	\textbf{Use case} & \textbf{U3.} Show Topic Pane\\
	\midrule
	\textbf{Goals} & G1\\
	\midrule
	\textbf{Requirements} & R17\\
	\midrule
	\textbf{Actor} & Farmer\\
	\midrule
	\textbf{Scenario} & 
	\begin{enumerate}[leftmargin=.4cm,noitemsep,topsep=0pt,before=\vspace{-3mm},after=\vspace{-4mm}]
		\item Sign in to the \textit{Farmer's WebApplication}.
		\item Click the \textit{Forum} tab.
	\end{enumerate}\\
	\midrule
	\textbf{Expected Output} & The \textit{Topic Page} containing a list of forum topics is displayed.\\
	\midrule
	\textbf{Test Result} & Success\\
	\bottomrule
	\caption{Test case verifying U3.}
\end{longtable}

\subsection{Show Production}

\begin{longtable}{@{}p{0.25\linewidth}p{0.71\linewidth}@{}}
	\toprule
	\textbf{Use case} & \textbf{U4.} Show Production\\
	\midrule
	\textbf{Goals} & G2\\
	\midrule
	\textbf{Requirements} & \textit{No valid requirement specified.}\\
	\midrule
	\textbf{Actor} & Farmer\\
	\midrule
	\textbf{Scenario} &
	\begin{enumerate}[leftmargin=.4cm,noitemsep,topsep=0pt,before=\vspace{-3mm},after=\vspace{-4mm}]
		\item Sign in to the \textit{Farmer's WebApplication}.
		\item Click the \textit{Production} tab.
	\end{enumerate}\\
	\midrule
	\textbf{Expected Output} & The \textit{Production Page} with farmer's production data is displayed.\\
	\midrule
	\textbf{Test Result} & Success\\
	\bottomrule
	\caption{Test case verifying U4.}
\end{longtable}

\subsection{Show Farm Info}

\begin{longtable}{@{}p{0.25\linewidth}p{0.71\linewidth}@{}}
	\toprule
	\textbf{Use case} & \textbf{U5.} Show Farm Info\\
	\midrule
	\textbf{Goals} & G6\\
	\midrule
	\textbf{Requirements} & \textit{No valid requirement specified.}\\
	\midrule
	\textbf{Actor} & Farmer\\
	\midrule
	\textbf{Scenario} &
	\begin{enumerate}[leftmargin=.4cm,noitemsep,topsep=0pt,before=\vspace{-3mm},after=\vspace{-4mm}]
		\item Sign in to the \textit{Farmer's WebApplication}.
		\item Click the \textit{Farm Info} tab.
	\end{enumerate}\\
	\midrule
	\textbf{Expected Output} & The \textit{Farm Info Page} containing soil, production, and sensor data is displayed.\\
	\midrule
	\textbf{Test Result} & \textbf{Failure}\\
	\bottomrule
	\caption{Test case verifying U5.}
\end{longtable}

\subsubsection*{Failure Description}

There is no tab called \textit{Farm Info} available. Such data can be seen inside \textit{Production} tab, nonetheless, it does not follow the use case description and no information regarding that change could be found in the provided documents.

\subsection{Send Request for Help}

\begin{longtable}{@{}p{0.25\linewidth}p{0.71\linewidth}@{}}
	\toprule
	\textbf{Use case} & \textbf{U6.} Send Request for Help\\
	\midrule
	\textbf{Goals} & G4\\
	\midrule
	\textbf{Requirements} & R11, R14\\
	\midrule
	\textbf{Actor} & Farmer\\
	\midrule
	\textbf{Scenario} &
	\begin{enumerate}[leftmargin=.4cm,noitemsep,topsep=0pt,before=\vspace{-3mm},after=\vspace{-4mm}]
		\item Sign in to the \textit{Farmer's WebApplication}.
		\item Click the \textit{Help} tab.
		\item Press the \textit{Create a new help request} button.
		\item Fill in the \textit{Subject} and \textit{Body} fields.
	\end{enumerate}\\
	\midrule
	\textbf{Expected Output} & A request is created and sent to the agronomist responsible for the given area.\\
	\midrule
	\textbf{Test Result} & Success\\
	\bottomrule
    \caption{Test case verifying U6.}
\end{longtable}


\subsubsection*{Unexpected Behavior}

The case when valid data is introduced works as expected. However, the system allows creating empty help requests and no input validation is in place.


\subsection{Sign Up}
\begin{longtable}{@{}p{0.25\linewidth}p{0.71\linewidth}@{}}
	\toprule
	\textbf{Use case} & \textbf{U7.} Sign Up\\
	\midrule
	\textbf{Goals} & G1 - G9\\
	\midrule
	\textbf{Requirements} & R24, R26, R27\\
	\midrule
	\textbf{Actor} & Farmer\\
	\midrule
	\textbf{Scenario} &
	\begin{enumerate}[leftmargin=.4cm,noitemsep,topsep=0pt,before=\vspace{-3mm},after=\vspace{-4mm}]
		\item Open \textit{Farmer's WebApplication}
		\item Click the \textit{Sign In} button.
		\item Click \textit{Don't have an account? Sign Up} button.
		\item Provide \textit{First Name, Last Name, Aadhaar, Telephone number, E-Mail, Password}.
		\item Click \textit{Sign Up} button.
 	\end{enumerate}\\
	\midrule
	\textbf{Expected Output} & An account is created.\\
	\midrule
	\textbf{Test Result} & Success\\
	\bottomrule
    \caption{Test case verifying U7.}
\end{longtable}

\subsubsection*{Unexpected Behavior}
The system correctly checks for missing fields and existing accounts. Agronomist receives a notification about a new farmer in the area.

However, the system does not ask user for the password confirmation, as described in the use case description in the RASD. Moreover, the system does not validate the format of the inserted data. 


\subsection{Login}
\begin{longtable}{@{}p{0.25\linewidth}p{0.71\linewidth}@{}}
	\toprule
	\textbf{Use case} & \textbf{U8.} Login Page\\
	\midrule
	\textbf{Goals} & G1 - G9\\
	\midrule
	\textbf{Requirements} & R24\\
	\midrule
	\textbf{Actor} & Farmer, Agronomist\\
	\midrule
	\textbf{Scenario} &
	\begin{enumerate}[leftmargin=.4cm,noitemsep,topsep=0pt,before=\vspace{-3mm},after=\vspace{-4mm}]
		\item Open \textit{Farmer's WebApplication}/\textit{Agronomist's WebApplication}.
		\item Click the \textit{Sign In} button.
		\item Provide \textit{E-Mail, Password}.
		\item Click the \textit{Sign In} button.
 	\end{enumerate}\\
	\midrule
	\textbf{Expected Output} & User is logged in and redirected to the home page.\\
	\midrule
	\textbf{Test Result} & Success\\
	\bottomrule
    \caption{Test case verifying U8.}
\end{longtable}

\subsubsection*{Additional Notes}
The case was tested and resulted in \textit{Success} on both clients' panels.


\subsection{Create Meeting}
\begin{longtable}{@{}p{0.25\linewidth}p{0.71\linewidth}@{}}
	\toprule
	\textbf{Use case} & \textbf{U9.} Create Meeting\\
	\midrule
	\textbf{Goals} & G7\\
	\midrule
	\textbf{Requirements} & R3, R4, R5, R6\\
	\midrule
	\textbf{Actor} & Agronomist\\
	\midrule
	\textbf{Scenario} &
	\begin{enumerate}[leftmargin=.4cm,noitemsep,topsep=0pt,before=\vspace{-3mm},after=\vspace{-4mm}]
		\item Sign in to the \textit{Agronomist's WebApplication}.
		\item Click on \textit{Farmers} in the navigation top bar.
		\item Click the button with three dots.
		\item Choose \textit{Plan a meeting} option.
		\item Insert \textit{Date, Start time, End time}.
		\item Click the \textit{OK} button.
 	\end{enumerate}\\
	\midrule
	\textbf{Expected Output} & A meeting is correctly created and displayed on the meeting list.\\
	\midrule
	\textbf{Test Result} & Success\\
	\bottomrule
    \caption{Test case verifying U9.}
\end{longtable}

\subsubsection*{Unexpected Behavior}
The system correctly checks for overlapping time periods during meeting creation, however allows meeting with the same start and end time. The system does not allow creating a meeting during \textit{working hours}, nonetheless their definition is not provided neither in the RASD nor in the DD. 


\subsection{Create Knowledge}
\begin{longtable}{@{}p{0.25\linewidth}p{0.71\linewidth}@{}}
	\toprule
	\textbf{Use case} & \textbf{U10.} Create Meeting\\
	\midrule
	\textbf{Goals} & G6\\
	\midrule
	\textbf{Requirements} & R19\\
	\midrule
	\textbf{Actor} & Agronomist\\
	\midrule
	\textbf{Scenario} &
	\begin{enumerate}[leftmargin=.4cm,noitemsep,topsep=0pt,before=\vspace{-3mm},after=\vspace{-4mm}]
		\item Sign in to the \textit{Agronomist's WebApplication}.
		\item Click on \textit{Knowledge} in the navigation top bar.
		\item Provide \textit{Title, Category, Description, Article}.
		\item Click the \textit{OK} button.
 	\end{enumerate}\\
	\midrule
	\textbf{Expected Output} & A Knowledge instance is correctly created.\\
	\midrule
	\textbf{Test Result} & Success\\
	\bottomrule
    \caption{Test case verifying U10.}
\end{longtable}


\subsection{Insert Farm Data}
\begin{longtable}{@{}p{0.25\linewidth}p{0.71\linewidth}@{}}
	\toprule
	\textbf{Use case} & \textbf{U11.} Insert Farm Data\\
	\midrule
	\textbf{Goals} & G7, G8\\
	\midrule
	\textbf{Requirements} & \textit{No valid requirement specified.}\\
	\midrule
	\textbf{Actor} & Farmer\\
	\midrule
	\textbf{Scenario} &
	\begin{enumerate}[leftmargin=.4cm,noitemsep,topsep=0pt,before=\vspace{-3mm},after=\vspace{-4mm}]
		\item Open \textit{Farmer's WebApplication}
		\item Click the \textit{Sign In} button.
		\item Click \textit{Don't have an account? Sign Up} button.
		\item Provide \textit{First Name, Last Name, Aadhaar, Telephone number, E-Mail, Password}.
		\item Click \textit{Sign Up} button.
		\item Provide \textit{Area, Address, Square kilometer}.
		\item Click \textit{Sign Up} button.
 	\end{enumerate}\\
	\midrule
	\textbf{Expected Output} & Farm data is updated, and the user is redirected to the home page.\\
	\midrule
	\textbf{Test Result} & Success\\
	\bottomrule
    \caption{Test case verifying U11.}
\end{longtable}

\subsubsection*{Unexpected Behavior}
Input format is not validated, which makes it possible to input negative number of farm's square kilometers.


\subsection{New Farmer Notification}
\begin{longtable}{@{}p{0.25\linewidth}p{0.71\linewidth}@{}}
	\toprule
	\textbf{Use case} & \textbf{U12.} Notification New Farmer\\
	\midrule
	\textbf{Goals} & G7\\
	\midrule
	\textbf{Requirements} & R26\\
	\midrule
	\textbf{Actor} & Farmer\\
	\midrule
	\textbf{Scenario} &
	\begin{enumerate}[leftmargin=.4cm,noitemsep,topsep=0pt,before=\vspace{-3mm},after=\vspace{-4mm}]
		\item Open \textit{Farmer's WebApplication}
		\item Click the \textit{Sign In} button.
		\item Click \textit{Don't have an account? Sign Up} button.
		\item Provide \textit{First Name, Last Name, Aadhaar, Telephone number, E-Mail, Password}.
		\item Click \textit{Sign Up} button.
		\item Provide \textit{Area, Address, Square kilometer}.
		\item Click \textit{Sign Up} button.
		\item Sign in to the \textit{Agronomist's WebApplication}.
 	\end{enumerate}\\
	\midrule
	\textbf{Expected Output} & There is a notification regarding a new farmer in the area in Agronomist's home page.\\
	\midrule
	\textbf{Test Result} & Success\\
	\bottomrule
    \caption{Test case verifying U12.}
\end{longtable}


\subsection{Create New Topic}

\begin{longtable}{@{}p{0.25\linewidth}p{0.71\linewidth}@{}}
	\toprule
	\textbf{Use case} & \textbf{U14.} Create New Topic \\
	\midrule
	\textbf{Goals} & G1\\
	\midrule
	\textbf{Requirements} & R15, R16 \\
	\midrule
	\textbf{Actor} & Farmer\\
	\midrule
	\textbf{Scenario} & \begin{enumerate}[leftmargin=.4cm,noitemsep,topsep=0pt,before=\vspace{-3mm},after=\vspace{-4mm}]
		\item Log in as a farmer to the \textit{Farmer's WebApplication}.
		\item Click on \textit{Forum} in the navigation top bar.
		\item Click on the \textit{Create new topic} button in the left bottom corner.
		\item Fill the body, tag and the topic.
		\item Click on the \textit{Create} button.
	\end{enumerate}\\
	\midrule
	\textbf{Expected Output} & A topic is successfully created and shown in the forum view.\\
	\midrule
	\textbf{Test Result} & Success\\
	\bottomrule
	\caption{Test case verifying U14.}
\end{longtable}

\subsubsection*{Unexpected Behavior}
It is possible to create a new topic with empty title, body, and without tags.

\subsection{Update Production Data}

\begin{longtable}{@{}p{0.25\linewidth}p{0.71\linewidth}@{}}
	\toprule
	\textbf{Use case} & \textbf{U15.} Update Production Data \\
	\midrule
	\textbf{Goals} & G2\\
	\midrule
	\textbf{Requirements} & \textit{No valid requirement specified.}\\
	\midrule
	\textbf{Actor} & Farmer\\
	\midrule
	\textbf{Scenario} & \begin{enumerate}[leftmargin=.4cm,noitemsep,topsep=0pt,before=\vspace{-3mm},after=\vspace{-4mm}]
		\item Log in as a farmer to the \textit{Farmer's WebApplication}.
		\item Click on the \textit{Production} in the navigation top bar.
		\item Click on the \textit{See more} button at the bottom of \textit{Production} box.
		\item Click on the pencil icon button present in the item of the production entries list that is to be edited.
		\item Change amount and note in the pop-up.
		\item Click on the \textit{OK} button.
	\end{enumerate}\\
	\midrule
	\textbf{Expected Output} & The production data is updated and new, updated values are presented in the production entries list.\\
	\midrule
	\textbf{Test Result} & Success\\
	\bottomrule
	\caption{Test case verifying U15.}
\end{longtable}


\subsubsection*{Unexpected Behavior}
Input format for the \textit{Amount} in the \textit{Edit the production} pop-up allows for inserting alphanumerical characters. Additionally, the inputs can be cleared and successfully saved. In both cases, the production data entry contains the \textit{Q.ta} field equal to zero and empty \textit{Note} field.


\subsection{Comment Topic}

\begin{longtable}{@{}p{0.25\linewidth}p{0.71\linewidth}@{}}
	\toprule
	\textbf{Use case} & \textbf{U16.} Comment Topic \\
	\midrule
	\textbf{Goals} & G1\\
	\midrule
	\textbf{Requirements} & R17\\
	\midrule
	\textbf{Actor} & Farmer\\
	\midrule
	\textbf{Scenario} & \begin{enumerate}[leftmargin=.4cm,noitemsep,topsep=0pt,before=\vspace{-3mm},after=\vspace{-4mm}]
		\item Log in as a farmer to the \textit{Farmer's WebApplication}.
		\item Click on \textit{Forum} in the navigation top bar.
		\item Click on the comment icon button in the left bottom corner of a topic.
		\item Write a comment in the text box.
		\item Click on the send icon button.
	\end{enumerate}\\
	\midrule
	\textbf{Expected Output} & The comment is saved in a topic and the number of topic's comment has been increased.\\
	\midrule
	\textbf{Test Result} & Success\\
	\bottomrule
	\caption{Test case verifying U16.}
\end{longtable}


\subsubsection*{Unexpected Behavior}
It is possible to add comment with empty content.

\subsection{Answer Help Request}

\begin{longtable}{@{}p{0.25\linewidth}p{0.71\linewidth}@{}}
	\toprule
	\textbf{Use case} & \textbf{U17.} Answer Help Request \\
	\midrule
	\textbf{Goals} & G1\\
	\midrule
	\textbf{Requirements} & R12\\
	\midrule
	\textbf{Actor} & Farmer\\
	\midrule
	\textbf{Scenario} & \begin{enumerate}[leftmargin=.4cm,noitemsep,topsep=0pt,before=\vspace{-3mm},after=\vspace{-4mm}]
		\item Log in as an agronomist to the \textit{Agronomist's WebApplication}.
		\item Click on \textit{Help requests} in the navigation top bar.
		\item Click on the comment icon button in the bottom right corner of a help request to be answered.
		\item Fill a text box with response.
		\item Click on the send icon button.
	\end{enumerate}\\
	\midrule
	\textbf{Expected Output} & The response is saved and visible in the farmer that is the help request's author.\\
	\midrule
	\textbf{Test Result} & Success\\
	\bottomrule
	\caption{Test case verifying U17.}
\end{longtable}


\subsubsection*{Unexpected Behavior}
\begin{itemize}
    \item It is possible to add a response with empty content.
    \item When the farmer tries to see the response, he selects the \textit{Forum} tab in the top navigation bar. The response is available in the \textit{No feed} tab, what is very counter-intuitive. Then, if the farmer answers again to the agronomist's response, the help request is considered as \textit{Closed} what is even more misleading. The names of help request's states does not reflect the states for described in the RASD, in the Figure 3.
\end{itemize}