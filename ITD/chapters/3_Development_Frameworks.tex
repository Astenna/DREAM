\chapter{Development Frameworks} \label{ch:development_frameworks}

\section{Programming Languages}

\subsection{TypeScript}
TypeScript is a strongly typed programming language build on JavaScript and maintained by Microsoft. According to Stack Overflow 2020 Developer survey, it was the second most loved programming language \cite{type_script}. Its several downsides and upsides are described below.

\textbf{Advantages}
\begin{itemize}
    \item Readability - thanks to strong typing, the code is more self-documenting, making it more suitable for large-scale projects.
    \item Early spotted bugs and types error - data types error are not only detected at runtime like in dynamic typing with JavaScipt, but also during compile time.
    \item Rich IDE support - information about types combined with IDEs support can produce a significant productivity boost, offering features like autocompletion, suggestions or just better code navigation.
\end{itemize}

\textbf{Disadvantages}
\begin{itemize}
    \item TypeScript is just a subset of JavaScript - experienced JavaScript developers may happen to run into some issues trying to use functions that are not available in TypeScript.
    \item Requires writing more code - especially in case of generic functions, the types definitions can look bloated and defined inline by default.
    \item Code needs to be transpiled into JavaScript - it can be done on the fly, but still can be a bottleneck for larger code bases.
\end{itemize}

\subsection{C\#}
C\# is a general purpose, strongly typed, object-oriented programming language designed by Microsoft. It runs on .NET execution system called the common language runtime (CLR) that not only provides runtime services, but also extensive libraries designed for building many kinds of apps \cite{c-sharp-tour}. Below, you can find some of its advantages and disadvantages.

\textbf{Advantages}
\begin{itemize}
    \item Active community - due to its popularity and various applications, an abundance of guidelines, tutorials, and documentation can be found on the Internet.
    \item Extensive tooling and libraries - developers can choose among many libraries available as \textit{nuget} packages and choose a convenient IDE well-suited for C\# development, such as \textit{Rider} or \textit{Visual Studio}  \cite{net-tools} \cite{nuget}.
    \item Open source and cross-platform - over the last years, C\# and .NET went through a huge transformation from proprietary, Windows-only to an open source, cross-platform technology. The change introduced significant performance boost and language improvements like implicit usings and new lambda capabilities \cite{net6}. 
\end{itemize}

\textbf{Disadvantages}
\begin{itemize}
    \item Stability issues for new releases - together with great transformations of C\# and .NET mentioned in the advantages list, some previously supported frameworks and functionalities were abandoned. In addition, the new releases of .NET Core introduced many breaking-changes \cite{net-breaking-changes}. 
    \item No standalone compiler, being tied to .NET's Common Language Runtime - the code can not be compiled into a machine code. Instead, it uses proprietary byte code that is executed on a virtual machine called \textit{Common Language Runtime} that needs to be available on the machine to run a C\# application \cite{c-sharp-tour}.
\end{itemize}

\subsection{Python}

Python is a high-level general-purpose programming language that is interpreted. With the usage of considerable indentation, its design philosophy prioritizes code readability. Its language elements and object-oriented approach are intended to assist programmers in writing clear, logical code for small and large-scale projects. Its advantages as well as drawbacks are presented below.

\textbf{Advantages}
\begin{itemize}
    \item High-level language that boosts productivity. When compared to other languages, the language features rich support libraries and clean object-oriented designs that enhance programmer efficiency by a significant margin.
    \item Interpreted language - Python runs the code immediately, line by line. In the event of an error, it suspends further execution and communicates the problem. Due to the fact that the language shows only one error at a time, it makes debugging simple.
    \item Portability - programming languages, such as C/C++, generate a need to rewrite the code in order to run the application on other platforms. Python, on the other hand, is not the same. Due to the usage of interpreter, code written in this language is platform-independent.

You should, however, take care not to add any system-dependent features.
\end{itemize}

\textbf{Disadvantages}
\begin{itemize}
    \item Execution speed - Python is a dynamically typed interpreted language. Because the language is interpreted, each line of code must be explicitly interpreted for execution. This is time-consuming, and hence slows down the execution process.
    \item Memory consumption - a necessary sacrifice for Python using dynamic typing is very high memory usage. It makes it practically unusable in scenarios when the program's performance plays an important role.
    \item Runtime errors - being a dynamically typed language means that the data type of variable can change any time. This may lead to unexpected runtime errors.
\end{itemize}

\section{Frameworks and libraries}
 \textit{React} \cite{react}, \textit{Ant Design} \cite{ant_design}, \textit{Redux} \cite{redux}

\subsection{Entity Framework}
Entity Framework is an open source, object-relational mapper for .NET that supports many types of databases. It provides support for LINQ queries, updates, schema migrations and change of tracking \cite{ef}\cite{ef-doc}. 

\textbf{Advantages}
\begin{itemize}
    \item No SQL knowledge necessary - the developers are not required to write SQL queries to interact with the database since Entity Framework takes the burden of translating the LINQ queries into SQL. 
    \item Defines a common syntax that can be used both for manipulating collections stored in application's memory or database, regardless of the database engine used.
\end{itemize}

\textbf{Disadvantages}
\begin{itemize}
    \item Possible lack of support for the latest or more sophisticated database engine features - entity framework may not immediately reflect the latest features added to database engines.
    \item Sometimes, lack of knowledge about the way how LINQ queries are translated may lead to creation of very expensive SQL queries.
\end{itemize}

\subsection{FluentValidation}
FluentValidation is a .NET library designed for definition of strongly-typed validation rules. It allows for setting up dedicated validator classes that define validation rules using fluent API \cite{fluentvalidation}.

\textbf{Advantages}
\begin{itemize}
    \item Easy creation of readable validation rules with meaningful error messages
    \item Decouples validation rules and models
    \item Can be added as a middleware that is executed even before the request with given model reaches controller
\end{itemize}
\textbf{Disadvantages}
\begin{itemize}
    \item Rules may be hard to debug
\end{itemize}

\subsection{AutoMapper}
AutoMapper is a library build for automating object-to-object mapping. The library aims at simplifying the task of mapping one object to another. It automatically maps fields with the same name leveraging reflection and for more complex cases it allows configuration of custom projections.

\textbf{Advantages}
\begin{itemize}
    \item Saves time automating mapping of fields with the same value
    \item Mappings configurations may be tested automatically using functions from the library
\end{itemize}
\textbf{Disadvantages}
\begin{itemize}
    \item Refactoring issues - after renaming a field that was used in the automatic mapping (name matching), the mapping will break without any compile time errors.
    \item Hard to debug
\end{itemize}

\subsection{Selenium}

\textit{Selenium WebDriver} \cite{selenium}

\todo{To describe: Mariusz}

\textbf{Advantages}
\begin{itemize}
    \item Test
\end{itemize}

\textbf{Disadvantages}
\begin{itemize}
    \item Test.
\end{itemize}

\subsection{Pytest}

\todo{To describe: Mariusz}

As a comprehensive Python testing tool, \textit{pytest} \cite{pytest} may be used for numerous sorts of software testing at various levels of abstraction. It is now one of the most widely used testing frameworks. It has sophisticated capabilities like'assert' rewriting, a third-party plugin model, and a powerful yet simple fixture model.

\textbf{Advantages}
\begin{itemize}
    \item Test
\end{itemize}

\textbf{Disadvantages}
\begin{itemize}
    \item Test.
\end{itemize}

\section{Middleware}
\todo{can we remove it? or, what do we describe here?}