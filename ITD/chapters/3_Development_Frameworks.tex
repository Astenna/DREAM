\chapter{Development Frameworks} \label{ch:development_frameworks}

This section of the article describes the programming languages and frameworks utilized throughout development, as well as the reasons behind their selection in forms of their advantages and disadvantages.

\section{Programming Languages}

\subsection{TypeScript}
TypeScript\cite{type_script} is a strongly typed programming language build on JavaScript and maintained by Microsoft. According to Stack Overflow 2020 Developer survey, it was the second most loved programming language. Its several downsides and upsides are described below.

\textbf{Advantages}
\begin{itemize}
    \item Readability - thanks to strong typing, the code is more self-documenting, making it more suitable for large-scale projects.
    \item Early spotted bugs and types error - data types error are not only detected at runtime like in dynamic typing with JavaScipt, but also during compile time.
    \item Rich IDE support - information about types combined with IDEs support can produce a significant productivity boost, offering features like autocompletion, suggestions or just better code navigation.
\end{itemize}

\textbf{Disadvantages}
\begin{itemize}
    \item TypeScript is just a subset of JavaScript - experienced JavaScript developers may happen to run into some issues trying to use functions that are not available in TypeScript.
    \item Requires writing more code - especially in case of generic functions, the types definitions can look bloated and defined inline by default.
    \item Code needs to be transpiled into JavaScript - it can be done on the fly, but still can be a bottleneck for larger code bases.
\end{itemize}

\subsection{C\#}
C\#\cite{c-sharp-tour} is a general purpose, strongly typed, object-oriented programming language designed by Microsoft. It runs on .NET execution system called the common language runtime (CLR) that not only provides runtime services, but also extensive libraries designed for building many kinds of apps . Below, you can find some of its advantages and disadvantages.

\textbf{Advantages}
\begin{itemize}
    \item Active community - due to its popularity and various applications, an abundance of guidelines, tutorials, and documentation can be found on the Internet.
    \item Extensive tooling and libraries - developers can choose among many libraries available as \textit{nuget}\cite{nuget} packages and choose a convenient IDE well-suited for C\# development, such as \textit{Rider} or \textit{Visual Studio}  \cite{net-tools}.
    \item Open source and cross-platform - over the last years, C\# and .NET went through a huge transformation from proprietary, Windows-only to an open source, cross-platform technology. The change introduced significant performance boost and language improvements like implicit usings and new lambda capabilities \cite{net6}. 
\end{itemize}

\textbf{Disadvantages}
\begin{itemize}
    \item Stability issues for new releases - together with great transformations of C\# and .NET mentioned in the advantages list, some previously supported frameworks and functionalities were abandoned. In addition, the new releases of .NET Core introduced many breaking-changes \cite{net-breaking-changes}. 
    \item No standalone compiler, being tied to .NET's Common Language Runtime - the code can not be compiled into a machine code. Instead, it uses proprietary byte code that is executed on a virtual machine called \textit{Common Language Runtime} that needs to be available on the machine to run a C\# application \cite{c-sharp-tour}.
\end{itemize}

\subsection{Python}

Python \cite{python} is a high-level general-purpose programming language that is interpreted. With the usage of considerable indentation, its design philosophy prioritizes code readability. Its language elements and object-oriented approach are intended to assist programmers in writing clear, logical code for small and large-scale projects. Its advantages as well as drawbacks are presented below.

\textbf{Advantages}
\begin{itemize}
    \item High-level language that boosts productivity. When compared to other languages, the language features rich support libraries and clean object-oriented designs that enhance programmer efficiency by a significant margin.
    \item Interpreted language - Python runs the code immediately, line by line. In the event of an error, it suspends further execution and communicates the problem. Due to the fact that the language shows only one error at a time, it makes debugging simple.
    \item Portability - programming languages, such as C/C++, generate a need to rewrite the code in order to run the application on other platforms. Python, on the other hand, is not the same. Due to the usage of interpreter, code written in this language is platform-independent.

\end{itemize}

\textbf{Disadvantages}
\begin{itemize}
    \item Execution speed - Python is a dynamically typed interpreted language. Because the language is interpreted, each line of code must be explicitly interpreted for execution. This is time-consuming, and hence slows down the execution process.
    \item Memory consumption - a necessary sacrifice for Python using dynamic typing is very high memory usage. It makes it practically unusable in scenarios when the program's performance plays an important role.
    \item Runtime errors - being a dynamically typed language means that the data type of variable can change any time. This may lead to unexpected runtime errors.
\end{itemize}

\section{Frameworks and libraries}

\subsection{Entity Framework}
Entity Framework\cite{ef} is an open source, object-relational mapper for .NET that supports many types of databases. It provides support for LINQ queries, updates, schema migrations and change of tracking \cite{ef-doc}. 

\textbf{Advantages}
\begin{itemize}
    \item No SQL knowledge necessary - the developers are not required to write SQL queries to interact with the database since Entity Framework takes the burden of translating the LINQ queries into SQL. 
    \item Defines a common syntax that can be used both for manipulating collections stored in application's memory or database, regardless of the database engine used.
\end{itemize}

\textbf{Disadvantages}
\begin{itemize}
    \item Possible lack of support for the latest or more sophisticated database engine features - entity framework may not immediately reflect the latest features added to database engines.
    \item Sometimes, lack of knowledge about the way how LINQ queries are translated may lead to creation of very expensive SQL queries.
\end{itemize}

\subsection{FluentValidation}
FluentValidation\cite{fluentvalidation} is a .NET library designed for definition of strongly-typed validation rules. It allows for setting up dedicated validator classes that define validation rules using fluent API.

\textbf{Advantages}
\begin{itemize}
    \item Easy creation of readable validation rules with meaningful error messages.
    \item Decouples validation rules and models.
    \item Can be added as a middleware that is executed even before the request with given model reaches controller.
\end{itemize}
\textbf{Disadvantages}
\begin{itemize}
    \item Rules may be hard to debug.
\end{itemize}

\subsection{AutoMapper}
AutoMapper\cite{automapper} is a library build for automating object-to-object mapping. The library aims at simplifying the task of mapping one object to another. It automatically maps fields with the same name leveraging reflection and for more complex cases it allows configuration of custom projections.

\textbf{Advantages}
\begin{itemize}
    \item Saves time automating mapping of fields with the same value.
    \item Mappings configurations may be tested automatically using functions from the library.
\end{itemize}
\textbf{Disadvantages}
\begin{itemize}
    \item Refactoring issues - after renaming a field that was used in the automatic mapping (name matching), the mapping will break without any compile time errors.
    \item Hard to debug.
\end{itemize}

\subsection{React}
React \cite{react} is a declarative, efficient, and flexible JavaScript/TypeScript library for building user interfaces created by Facebook. It lets the user compose complex UIs from small and isolated pieces of code called “components”.

\textbf{Advantages}
\begin{itemize}
    \item React allows users to reuse components, which saves time and readability.
    \item Code is easier to maintain and more flexible due to its modular structure.
    \item Improved performance due to virtual DOM.
\end{itemize}
\textbf{Disadvantages}
\begin{itemize}
    \item High pace of development – environment continually changes.
\end{itemize}

\subsection{Ant Design}
Ant Design \cite{ant_design} is an open source React UI library written in TypeScript. It comes with a set of high-quality React components and theme customization capability.

\textbf{Advantages}
\begin{itemize}
    \item Polished look and feel.
    \item Built using static typing.
    \item Excellent form handling.
\end{itemize}
\textbf{Disadvantages}
\begin{itemize}
    \item High pace of development.
\end{itemize}

\subsection{Redux}
Redux \cite{redux} is an open-source JavaScript library for managing and centralizing application state. It has a simple, limited API designed to be a predictable container for application state.

\textbf{Advantages}
\begin{itemize}
    \item Predictable state, which makes it possible to implement tasks like undo/redo.
    \item Easy to debug and test.
\end{itemize}
\textbf{Disadvantages}
\begin{itemize}
    \item Requires a lot of boilerplate code.
\end{itemize}

\subsection{Selenium WebDriver} \label{subsec:selenium_webdriver}

\textit{Selenium WebDriver} \cite{selenium} is one of the most popular development tools utilized in Web UI automation. It has no direct interaction with the web components on a webpage. A browser-specific driver serves as a conduit between the test script and the web browser. Then, Selenium locators are used to locate components on a page so that relevant methods for interacting with the element may be employed.

\textbf{Advantages}
\begin{itemize}
    \item Support for a variety of programming languages - offers native bindings for JavaScript, Python, Java, C\#, Ruby, and Kotlin.
    \item Platform independent - compatible with Windows, Linux, macOS, Android (with Selendroid Appium or Robotium), and iOS (with iOS-driver or Appium).
    \item Cross-browser - supports all the major browsers, such as Chrome/Chromium, Safari, Opera, Firefox, and Edge.
\end{itemize}

\textbf{Disadvantages}
\begin{itemize}
    \item Only used for web-based apps - it cannot be used to automate desktop application testing since it cannot recognize desktop app objects. It is solely intended for testing web applications with the browsers listed above.
    \item High test maintenance - the framework's reliance on a single, rigid element identifier may lead to fragile tests. In a situation when an element's identifier in the user interface changes, the tests may break.
    \item Steep learning curve - requires knowledge and understanding  of at least one of aforementioned programming languages. It does not allow for codeless testing, as offered by other existing tools.
\end{itemize}

\subsection{pytest}\label{subsec:pytest}

As a comprehensive Python testing tool, \textit{pytest} \cite{pytest} may be used for numerous sorts of software testing at various levels of abstraction. It is now one of the most widely used testing frameworks. It has sophisticated capabilities like "assert" rewriting, a third-party plugin model, and a powerful yet simple fixture model.

\textbf{Advantages}
\begin{itemize}
    \item Flexible and easy to use fixtures allow for test parametrization and help minimizing the boilerplate code.
    \item Extensible - pytest can be effortlessly extended with a variety of hooks and plugins. They could enhance its functionalities by introducing parallel test execution or just explicitly marking dependencies and the order in which the test cases should be run.
    \item Compact test suites - pytest established the idea for the tests to be simple Python functions, rather than forcing developers to incorporate their tests within major test classes.
    \item Can be easily integrated with \textit{Selenium WebDriver}, described in section \ref{subsec:selenium_webdriver}.
\end{itemize}

\textbf{Disadvantages}
\begin{itemize}
    \item Incompatibility with other testing frameworks - because pytest requires tests to be written using its own proprietary techniques, it unavoidably sacrifices convenience for compatibility. In other words, building tests for pytest binds the programmer to it, and the only way to use another testing framework is to rewrite the majority of the code.
\end{itemize}

\section{Relational database management system}
\subsection{Postgres}
Postgres, or PostgreSQL\cite{postgresql}, is a reliable, free and open source database management system with over 30 years of active development and strong community.

\textbf{Advantages}
\begin{itemize}
    \item Comprehensive documentation and active community.
    \item No license purchase required.
    \item Support for geographic objects, enabling use as a geospatial data store.
\end{itemize}

\textbf{Disadvantages}
\begin{itemize}
    \item Major version updates may require long downtimes for large databases.
    \item High deployment cost on major cloud providers (e.g., Microsoft Azure).
\end{itemize}