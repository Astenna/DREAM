\chapter{Development Frameworks} \label{ch:development_frameworks}

\section{Programming Languages}

\textit{TypeScript} \cite{type_script}, \textit{React Native} \cite{react_native}, \textit{Ant Design} \cite{ant_design}, \textit{Redux} \cite{redux}

\subsection{C\#}
C\# is a general purpose, strongly typed, object-oriented programming language designed by Microsoft. It runs on .NET execution system called the common language runtime (CLR) that not only provides runtime services, but also extensive libraries designed for building many kinds of apps \cite{c-sharp-tour}. Below, you can find some of its advantages and disadvantages .

\textbf{Advantages}
\begin{itemize}
    \item Active community - due to its popularity and various applications, an abundance of guidelines, tutorials, and documentation can be found on the Internet.
    \item Extensive tooling and libraries - developers can choose among many libraries available as \textit{nuget} packages and choose a convenient IDE well-suited for C\# development, such as \textit{Rider} or \textit{Visual Studio} \cite{nuget} \cite{net-tools}.
    \item Open source and cross-platform - over the last years, C\# and .NET went through a huge transformation from proprietary, Windows-only to an open source, cross-platform technology. The change introduced significant performance boost and language improvements like implicit usings and new lambda capabilities \cite{net6}. 
\end{itemize}

\textbf{Disadvantages}
\begin{itemize}
    \item Stability issues for new releases - together with great transformations of C\# and .NET mentioned in the advantages list, some previously supported frameworks and functionalities were abandoned. In addition, the new releases of .NET Core introduced many breaking-changes \cite{net-breaking-changes}. 
    \item No standalone compiler, being tied to .NET's Common Language Runtime - the code can not be compiled into a machine code. Instead, it uses proprietary byte code that is executed on a virtual machine called \textit{Common Language Runtime} that needs to be available on the machine to run a C\# application \cite{c-sharp-tour}.
    \item Tied to .NET and Microsoft - the developers do not have the full control over the language, and for some people the relation with Microsoft is a disadvantage, since they dislike Microsoft \todo{is it OK? or too opinion-based?}.
\end{itemize}

\section{Frameworks}

\section{Middleware}
