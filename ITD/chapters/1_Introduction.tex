\chapter{Introduction}

\section{Purpose}

The purpose of this document is to present the work done during the implementation of the Data-dRiven PrEdictive FArMing in Telangana (\textit{DREAM}) application's initial prototype. It follows the directives and instructions specified in the \textit{I\&T assignment goal, schedule, and rules} \cite{reference_doc2}. A thorough backdrop together with problem definition may be found in the \textit{Requirement Engineering and Design Project: goal, schedule, and rules} document \cite{reference_doc}.

\section{Scope}

Implementation and Test Deliverable (\textit{ITD}) provides an overview of the languages and frameworks utilized during development, as well as a synopsis of the functionalities accessible in the current version of the program.

Furthermore, described here are: the created code, a document outlining the structure of the code, the development frameworks used, installation instructions, and information on how software tests were performed.

This document, together with the Requirement Analysis and Specification Document (\textit{RASD}) \cite{rasd} and the Design Document (\textit{DD}) \cite{dd}, is intended to inform on the completion of the \textit{DREAM} application.

\section{Definitions and Acronyms}

\subsection{Definitions}

\begin{center}
    \begin{longtable}{@{}p{0.28\linewidth} p{0.68\linewidth}@{}}
		\toprule
		\textbf{Expression}     & \textbf{Definition}\\
		
		\midrule
        Data transfer object & Object that encapsulates data. Used to transport data between the API layer and Business logic layer. \\
        Database migration & A feature of Entity Framework that allows for incremental changes of database schema as well as definition of the database schema based solely on the classes defined in the code. \\
        Web driver & A remote control interface that allows for user agent introspection and control. It provides a platform- and language-neutral wire protocol that allows out-of-process applications to remotely command web browser behavior.\\
	\bottomrule
	\end{longtable}
\end{center}

\subsection{Acronyms}

\begin{center}
	\begin{longtable}{@{}p{0.28\linewidth} p{0.68\linewidth}@{}}
		\toprule
		\textbf{Acronyms}   & \textbf{Expression}\\
		\endfirsthead
		\midrule
		API                 & Application PRogramming Interface\\
		DD                  & Design Document\\
		DOM                 & Document Object Model\\
		DREAM               & Data-dRiven PrEdictive FArMing in Telangana\\
		ITD					& Implementation and Test Deliverable\\
		JWT                 & JSON Web Token\\
		LINQ                & Language-Integrated Query\\
		R                   & Requirement\\
		RASD                & Requirements Analysis and Specifications Document\\
		SQL                 & Structured Query Language\\
		UI                  & User Interface\\
		\bottomrule
	\end{longtable}
\end{center}

\section{Revision History}

\begin{center}
	\begin{longtable}{@{}p{0.18\linewidth} p{0.18\linewidth} p{0.57\linewidth}@{}}
		\toprule
		\textbf{Date}   & \textbf{Revision} & \textbf{Notes}\\
		\midrule
        06.02.2022     	& v.1.0             & First release.\\
% 		\date{} & v.1.1 & \begin{itemize}[label={--},leftmargin=.4cm,noitemsep,topsep=0pt,before=\vspace{-3.5mm},after=\vspace{-4mm}]
% 			\item Change 1
% 			\item Change 2
% 		\end{itemize}\\
		\bottomrule
	\end{longtable}
\end{center}

\printbibliography[title={Reference Documents}, keyword=intro, heading=subbibnumbered]

\section{Document Structure}

\begin{enumerate}
    \item \textbf{Introduction:} describes the main aim of the work as well as its scope. It also specifies specific definitions, acronyms, and abbreviations that will be used throughout the text.
    \item \textbf{Product Functions:} discusses the requirements and functionalities that are actually implemented in the initial prototype of the program, as well as the reasoning for including them and rejecting others.
    \item \textbf{Development Frameworks:} explains the languages and frameworks used during development, together with motivation for their choice. Furthermore, it provides information about the adopted middleware and any API employed that was not included in the DD.
    \item \textbf{Code Structure:} elaborates on the structure of the code, including the organization of the different modules and the use of the different programming languages.
    \item \textbf{Testing:} describes the testing procedures, the key test cases, as well as the results of performed tests.
    \item \textbf{Installation Guide:} provides instructions on how to install the program and how to run it. It lists all the prerequisites that needs to be satisfied to run the application successfully.
    \item \textbf{Effort Spent:} gives a detailed summary of the time spent during the development of the program by each contributor.
    \item \textbf{References}
\end{enumerate}
