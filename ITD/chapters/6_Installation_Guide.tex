\chapter{Installation Guide} \label{ch:installation_guide}

This section contains instructions for installing and running the application successfully. It specifies all the conditions that must be met in order for the program to execute correctly.

The simplest way to run the application with just a single, simple command is by using docker compose. The prerequisites and the simple command is described in the section \ref{sec:docker}. 

Moreover, the application can also be run without docker containers. In that case, however, the installation steps are much longer and complex. It requires: 
\begin{itemize}
    \item Postgres database installation (recommended version: 14.1)
    \begin{itemize}
        \item Configured to accept requests at port 5432.
        \item If necessary, adjust the connection string in \textit{Code\slash Server\slash API\slash appsettings.json} (the Server address in the \textit{appsettings} should be set to 127.0.0.1 instead of \textit{postgres} as in the docker approach).
    \end{itemize}
    \item Following the steps described in \ref{sec:server} for Server installation.
    \item Following the steps described in \ref{sec:client} for Client installation.
\end{itemize}

Depending on the installation approach choice, the client application will be available on:
\begin{itemize}
    \item Port 1337 for the installation with docker containers.
    \item Port 3000 for the installation without docker containers.
\end{itemize}

\section{Docker} \label{sec:docker}
Before executing the command depicted in \ref{lst:docker_compose} install the latest Docker Desktop version from the \href{https://www.docker.com/products/docker-desktop}{official website}. The minimum version of docker required is 4.4.4 (73704). After successful installation, open a terminal in the \textit{Code} directory and run the command from the listing \ref{lst:docker_compose}.\\

\begin{lstlisting}[language=bash, caption={Docker compose.}, label=lst:docker_compose]
docker-compose up -d --build
\end{lstlisting}

\section{Server}\label{sec:server}
The only prerequisite is installation of the .NET 6.0 SDK. Then, the installation requires execution of commands in a terminal opened inside the root project's directory presented in listings: \ref{lst:api_deps_and_build}, \ref{lst:business_logic_deps_and_build}, \ref{lst:data_access_deps_and_build}, \ref{lst:publish_server}. It is important to preserve the order of listings. 


\begin{lstlisting}[language=bash, caption={Install dependencies and build \textit{API}.}, label=lst:api_deps_and_build]
cd Code/Server/API/
dotnet restore
dotnet build --configuration Release --no-restore
\end{lstlisting}

\begin{lstlisting}[language=bash, caption={Install dependencies and build \textit{BusinessLogic}.}, label=lst:business_logic_deps_and_build]
cd Code/Server/BusinessLogic/
dotnet restore
dotnet build --configuration Release --no-restore
\end{lstlisting}

\begin{lstlisting}[language=bash, caption={Install dependencies and build \textit{DataAccess}.}, label=lst:data_access_deps_and_build]
cd Code/Server/DataAccess/
dotnet build --configuration Release --no-restore
\end{lstlisting}

\begin{lstlisting}[language=bash, caption={Publish and run server application.}, label=lst:publish_server]
cd Code/Server/
dotnet publish --configuration Release
cd ./API/bin/Release/net6.0/publish/
dotnet run ./API.dll
\end{lstlisting}

\section{Client}\label{sec:client}
Before executing npm commands, it is necessary to install node.js instance. Recommended versions: 14.17 or 16.13. Afterwards, open terminal in the root directory of the project, run the npm commands from the listing \ref{lst:client_deps_and_build} should be executed.\\

\begin{lstlisting}[language=bash, caption={Install dependencies and build \textit{Client}.}, label=lst:client_deps_and_build]
cd Code/Client/
npm install
npm start
\end{lstlisting}

\section{Running System Tests}

The whole system testing process requires downloading and installation of \href{https://www.python.org/downloads/release/python-3100/}{Python 3.10}. This is the version used throughout the development process of the initial prototype of the system.

The second step is to install all the necessary packages. Those are listed inside \textit{requirements.txt} file. See listing \ref{lst:requirements} for specific commands to execute.

\begin{lstlisting}[language=bash, caption={Install requirements.}, label=lst:requirements]
cd Code/system_tests/
pip install requirements -r requirements.txt
\end{lstlisting}

Additionally, since \textit{Selenium WebDriver} is employed, it is necessary to install a web browser. The one used in system tests is \textit{Google Chrome}. Its latest stable release could be downloaded via the \href{https://www.google.com/intl/en/chrome/}{official website} or a package manager as shown in the listing \ref{lst:install_chrome}. Throughout the development process, system tests were executed in the headless mode (without visible browser's UI shell). This could be changed inside \textit{conftest.py} configuration file.

\begin{lstlisting}[language=bash, caption={Install Google Chrome.}, label={lst:install_chrome}]
sudo apt install google-chrome-stable
\end{lstlisting}

Finally, after all the previous steps are completed, it is possible to run the system tests using \textit{pytest} as depicted in the listing \ref{lst:run_pytest}. If it is needed to run only a particular subset of available test cases, please refer to \href{https://docs.pytest.org/en/latest/how-to/usage.html}{pytest documentation}.

\begin{lstlisting}[language=bash, caption={Run system tests.}, label={lst:run_pytest}]
pytest
\end{lstlisting}